\chapter{Implementation}

\section{Algorithm implementation}
\label{sec:algorithm_impl}

The input variables and output of the implementation can be found in \fullref{sec:decision_algorithm}. This section will cover the entire algorithm, getting from the input variables to the output.

This function is written in Julia, making use of its ability to use Unicode characters in variable names, allowing me to use the same mathematical notation in the code as in the design documentation. The following is an extract from the source code of the framework (\inline{/src/covert\_channels/covert\_channels.jl})

\begin{lstlisting}[language=JuliaLocal, style=julia]
function method_calculations(covert_methods::Vector{covert_method}, env::Dict{Symbol, Any}, Eₚ::Vector{Int64}=[], current_method::Int64=0)::NTuple{2, Vector{Float64}}
    # Get the queue data
    q = get_queue_data(env[:queue])

    # Covert score, higher is better : Method i score = scores[i]
    S = zeros(Float64, length(covert_methods))
    # Rate at which to send covert packets : Method i rate = rates[i]
    R = zeros(Float64, length(covert_methods))
    
    if isempty(q)
        @error "No packets in queue, cannot determine method" q
        return S, R
    end
    
    #@warn "Hardcoded response to determine_method"
    L = [get_layer_stats(q, Layer_type(i)) for i ∈ 2:4]

    # Eₗ : Environment length : Number of packets in queue
    Eₗ = length(q)

    # Eᵣ : Environment rate : (Packets / second)
    Eᵣ = Eₗ / abs(last(q).cap_header.timestamp - first(q).cap_header.timestamp)

    # Eₛ : Environment desired secrecy : User supplied (Default: 5)
    Eₛ = env[:desired_secrecy]

    # Get the count of hosts local to the supplied ip (we don't want to consider external ones).
    Eₕ = get_local_host_count(q, env[:dest_ip])

    for (i, method) ∈ enumerate(covert_methods)
        Lᵢ_temp = filter(x -> method.type ∈ keys(x), L)
        if isempty(Lᵢ_temp)
            @warn "No packets with valid headers" method.type L
            continue
        end
        # Lᵢ : the layer that method i exists on
        Lᵢ = Lᵢ_temp[1]

        # Lₛ : The sum of packets that have a valid header in Lᵢ
        Lₛ = +(collect(values(Lᵢ))...)

        # Lₚ : Percentage of total traffic that this layer makes up
        Lₚ = Lₛ / Eₗ

        # Pᵢ is the percentage of traffic 
        Pᵢ = Lₚ * (Lᵢ[method.type] / Lₛ)

        # Bᵢ is the bit capacity of method i
        Bᵢ = method.payload_size

        # Cᵢ is the penalty / bonus for the covertness
        #  has bounds [0, 2] -> 0% to 200% (± 100%)
        Cᵢ = 1 - ((method.covertness - Eₛ) / 10)

        # Score for method i
        #  Pᵢ * Bᵢ : Covert bits / Environment bits
        #  then weight by covertness
        #@info "S[i]" Pᵢ Bᵢ Cᵢ Pᵢ * Bᵢ * Cᵢ
        S[i] = Pᵢ * Bᵢ * Cᵢ

        # Rate for method i
        #  Eᵣ * Pᵢ : Usable header packets / second
        #  If we used this much it would be +100% of the environment rate, so we scale it down
        #  by dividing by hosts on the network, Eₕ.
        #  then weight by covertness
        #  We don't want to go over the environment rate, so reshape covertness is between [0, 1] (1 being 100% of env rate)
        #  (Eᵣ * Pᵢ * (Cᵢ / 2)) / Eₕ : Rate of covert packets / second
        #  ∴ 1 / Eᵣ * Pᵢ * (Cᵢ / 2) : Interval between covert packets
        #@info "R[i]" Eₕ Eᵣ Pᵢ Cᵢ/2 Eₕ / (Eᵣ * Pᵢ * (Cᵢ / 2))
        R[i] = Eₕ  / (Eᵣ * Pᵢ * (Cᵢ / 2)) 
    end

    # Eₚ (arg) : Environment penalty : Penalty for failing to work previously
    for i ∈ Eₚ
        S[i] *= 0.1 # 10% of original score
    end

    # Allow for no current method (as the case is when recovering)
    current_method != 0 && (S[current_method] *= 1.1) # Encourage current method (+10%)

    return S, R
end
\end{lstlisting}

The \inline{method\_calculations} function only does the calculations, not the selection of the best method, which is done by a function that wraps this one and returns an index $i$ and a rate $R_i$ for the highest scoring method ($S_i$).

\section{Integrity check}
