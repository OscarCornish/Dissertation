\begin{abstract}
    Adaptive covert communication systems are those that adapt to their surrounding environment to improve their covertness and reliability. The majority of current covert system implementations use a single covert channel to communicate. However, this approach leaves covert systems with a single point of failure, and prone to detection when they don't match the environment. In this paper, I propose a framework for an adaptive covert communication system, that can use a variety of covert channels to communicate, so that it can adapt to the environment and recover from failing channels. The framework is evaluated against a set of objectives and is shown to be effective at adapting to the environment and recovering from failing channels when tested in an isolated environment. Throughout the paper, I propose a few standards for covert communication systems, such as a covert padding technique and a compression technique for covert protocols. While the framework provides a good foundation for adaptive covert communication systems and does make the detection of covert communications harder for an adversary, there is still plenty of work to be done before it can be used effectively in a real-world scenario. \\
    This project aligns with the following CyBok skill \textbf{Network Security}
    \textbf{Keywords:} Adaptive Covert Communication Channels, Digital steganography, TCP/IP stack
\end{abstract}

