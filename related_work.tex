\chapter{Related Work}
\label{ch:related_work}

\section{Towards Adaptive Covert Communication Systems}

Most proposed covert systems are based on a single underlying channel, this is primarily due to the nature of academic work, however these channels are not work well in real-world scenarios because they are 'reliant on a particular method' \cite{TWACCS}.

There is a solution to this issue, a covert system that is built upon multiple underlying covert channels, that is adaptable and flexible to the dynamic changes in the network environment \cite{TWACCS}. \cite{TWACCS} proposes a framework that does this, it utilises a group of covert channels and operates in two phases; The first phase is a learning phase that identifies which of the channels are applicable to the current environment, this phase is continuous to allow the system to adapt to live changes in the environment. The other phase uses these channels to transfer data between the sender and receiver.

This approach to covert systems has a few benefits, as stated it provides redundancy \& reliability \cite{TWACCS} to the communication over a static channel approaches, providing protections communication failures that can arise from interference or from preventative measures such as firewalls, active wardens, and protocol normalisation. Not only is this type of system harder to prevent, it is harder to detect, its adaption to the network means anomaly based systems are rendered ineffective, and the irregular and undefined behaviour of the system makes it difficult to detect using signatiure based systems \cite{TWACCS}.

These additional benefits come at a cost, addtional data to transmit, which can mean additional packets, and thus a lower quality of covertness. There is also however a covertness benefit that comes from the unpredictable nature of stego-objects, that makes more difficult to distinguish from noise \cite{ECopSSUOCC}.

Unfortunately, the link provided by paper is dead, and unarchived, so the exact details of the implemented system is unknown, meaning i will not be able to build on the work codebase of this paper.

\section{Dynamic Routing in Covert Channel Overlays Based on Control Protocols}
\label{sec:DRiCCBoCP}

Inorder to effectively communicate with reliability and assured integrity, a protocol is required. Implementing protocols however requires space, which is not in abundance in covert channels. \cite{DRiCCBoCP} proposes a dynamic sized "micro-protocol", these MP's can be dynamic because of how information about packets and payloads is conveyed. Normal network headers, like those defined in \cite{rfc791} \& \cite{Trfc793}, are designed to work in a "contextless" way, a single packet describes where it is coming to and from, and information about the payload. This is not the case for covert channels, since there are only two parties involved, they can use a "contextful" approach, where the information regarding a communication channel can be managed by the endpoints of the channel as opposed to the data. This allows the microprotocols to be reduced to context updates, and the majority of payloads can be data only (with a small header to declare that it is data). For static channels there is no benefit to microprotocols \cite{DRiCCBoCP}, due to their unchanging context.

This paper also proposes a dynamic underlying protocol change \cite{DRiCCBoCP}, that abstracts the underlying covert channel through the use of an API this allows for different covert channels to be used with no change to the main logic of the program, this also allows channels of be implemented regardless of classification \cite{DRiCCBoCP}, permitting both covert timing channels and covert storage channels.
