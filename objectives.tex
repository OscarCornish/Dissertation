\chapter{Project objectives}
\label{ch:objectives}

This paper seeks to propose and evaluate a framework for an adaptive covert communication system, drawing on prior work in the field, outlined in \fullref{ch:background} and \fullref{ch:related_work}. The framework will be evaluated against the following objectives, using the terms \textit{\textbf{MUST}}, \textit{\textbf{SHOULD}} and \textit{\textbf{MAY}} to describe the importance of each objective, as defined in \cite{rfc2119}:

\begin{enumerate}
    \item \textit{\textbf{MUST}} be able to determine and switch to the best communication channel for the current situation.
    \item \textit{\textbf{MUST}} be able to detect and recover from failures in communication.
    \item \textit{\textbf{MUST}} be able to adapt to changes in the environment. 
    \item \textit{\textbf{MUST}} Encrypt and decrypt messages using a shared secret key.
    \item \textit{\textbf{SHOULD}} be able to recover from the complete failure of a communication channel.
    \item \textit{\textbf{MAY}} Allow bidirectional communication.
    \item \textit{\textbf{MAY}} Handle multiple channels at a time.
\end{enumerate}

I will also evaluate the framework against the following non-functional requirements:

\begin{enumerate}
    \item The effect on the covertness of the framework that comes as a result of its adaptive nature.
    \item The applicability of the framework to real-world scenarios.
    \item The capability to add new communication channels with reasonable simplicity.
\end{enumerate}

